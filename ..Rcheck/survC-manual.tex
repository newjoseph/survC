\nonstopmode{}
\documentclass[letterpaper]{book}
\usepackage[times,inconsolata,hyper]{Rd}
\usepackage{makeidx}
\makeatletter\@ifl@t@r\fmtversion{2018/04/01}{}{\usepackage[utf8]{inputenc}}\makeatother
% \usepackage{graphicx} % @USE GRAPHICX@
\makeindex{}
\begin{document}
\chapter*{}
\begin{center}
{\textbf{\huge Package `survC'}}
\par\bigskip{\large \today}
\end{center}
\ifthenelse{\boolean{Rd@use@hyper}}{\hypersetup{pdftitle = {survC: Survival Model Validation Utilities}}}{}
\ifthenelse{\boolean{Rd@use@hyper}}{\hypersetup{pdfauthor = {Minhyuk Kim}}}{}
\begin{description}
\raggedright{}
\item[Title]\AsIs{Survival Model Validation Utilities}
\item[Version]\AsIs{0.1.0}
\item[Author]\AsIs{Minhyuk Kim [aut, cre]}
\item[Maintainer]\AsIs{Minhyuk Kim }\email{mhkim@zarathu.com}\AsIs{}
\item[Description]\AsIs{Provides helper functions to compute linear predictors, time-dependent ROC curves,
and Harrell's concordance index for Cox proportional hazards models, alongside
data-preparation utilities for ADPKD validation workflows.}
\item[Imports]\AsIs{survival, stats, timeROC, officer, rvg, data.table, readxl}
\item[License]\AsIs{MIT + file LICENSE}
\item[Encoding]\AsIs{UTF-8}
\item[Roxygen]\AsIs{list(markdown = TRUE)}
\item[RoxygenNote]\AsIs{7.3.2}
\item[Suggests]\AsIs{magrittr, jstable, testthat (>= 3.0.0)}
\item[Config/testthat/edition]\AsIs{3}
\end{description}
\Rdcontents{Contents}
\HeaderA{calc\_risk\_score}{Compute risk scores from a fitted survival model}{calc.Rul.risk.Rul.score}
%
\begin{Description}
This helper wraps \code{stats::predict()} for \code{coxph} objects so that package users
can easily obtain linear predictors (default) or risk scores to feed into
downstream metrics such as time-dependent ROC or Harrell's C-index.
\end{Description}
%
\begin{Usage}
\begin{verbatim}
calc_risk_score(model, data = NULL, type = "lp", ...)
\end{verbatim}
\end{Usage}
%
\begin{Arguments}
\begin{ldescription}
\item[\code{model}] A fitted \code{coxph} object.

\item[\code{data}] Optional dataset on which to score the model. Defaults to the
training data stored within \code{model}.

\item[\code{type}] Scale of the predictions to return. Either \code{"lp"} (linear
predictor, the default) or \code{"risk"}. If \code{NULL} or omitted, \code{"lp"} is used.

\item[\code{...}] Additional arguments passed to \code{\LinkA{stats::predict()}{stats::predict()}}.
\end{ldescription}
\end{Arguments}
%
\begin{Value}
A numeric vector containing the requested risk scores.
\end{Value}
%
\begin{Examples}
\begin{ExampleCode}
if (requireNamespace("survival", quietly = TRUE)) {
  fit <- survival::coxph(survival::Surv(time, status) ~ age, data = survival::lung)
  # Linear predictor on the training data
  calc_risk_score(fit)

  # Risk scale predictions on new data
  calc_risk_score(fit, survival::lung, type = "risk")
}
\end{ExampleCode}
\end{Examples}
\HeaderA{cindex\_calc}{Calculate Harrell's C-index with 95\% CI}{cindex.Rul.calc}
%
\begin{Description}
Calculate Harrell's C-index with 95\% CI
\end{Description}
%
\begin{Usage}
\begin{verbatim}
cindex_calc(model, newdata = NULL, digits = 3)
\end{verbatim}
\end{Usage}
%
\begin{Arguments}
\begin{ldescription}
\item[\code{model}] a 'coxph' object

\item[\code{newdata}] optional validation dataset

\item[\code{digits}] number of decimal places for rounding (default 3).
\end{ldescription}
\end{Arguments}
%
\begin{Value}
numeric vector of C-index (lower, upper)
\end{Value}
%
\begin{Examples}
\begin{ExampleCode}
library(survival)
fit <- coxph(Surv(time, status) ~ age + sex, data = lung)
cindex_calc(fit)



\end{ExampleCode}
\end{Examples}
\HeaderA{prepare\_adpkd\_dataset}{Prepare ADPKD cohort data for survival modelling}{prepare.Rul.adpkd.Rul.dataset}
%
\begin{Description}
This helper reproduces the preprocessing pipeline that was previously in
\code{temp.R}. It reads the baseline and follow-up Excel files, performs column
harmonisation, derives follow-up times and laboratory summaries, and returns a
cleaned dataset split into training and validation subsets.
\end{Description}
%
\begin{Usage}
\begin{verbatim}
prepare_adpkd_dataset(
  baseline_path,
  followup_path,
  followup_reference = as.Date("2025-08-01"),
  train_size = 300,
  seed = 123456L
)
\end{verbatim}
\end{Usage}
%
\begin{Arguments}
\begin{ldescription}
\item[\code{baseline\_path}] Path to the baseline Excel file (sheet 1, skip = 1).

\item[\code{followup\_path}] Path to the follow-up Excel file (sheet 1).

\item[\code{followup\_reference}] Date used when the RRT start date is missing. Either
a \code{Date} or something coercible via \code{\LinkA{as.Date()}{as.Date}}. Defaults to
"2025-08-01".

\item[\code{train\_size}] Number of subjects to sample into the training set. If this
exceeds the number of rows it falls back to \code{nrow(data)}.

\item[\code{seed}] Integer seed for the train/validation split (default \code{123456}).
\end{ldescription}
\end{Arguments}
%
\begin{Value}
A list with elements \code{data} (cleaned dataset), \code{train}, \code{validation},
\code{varlist}, and \code{labels} (if \code{jstable} is available). All datasets are
returned as \code{data.table} objects.
\end{Value}
\HeaderA{tdroc\_calc}{Calculate time-dependent ROC and AUC}{tdroc.Rul.calc}
%
\begin{Description}
Calculate time-dependent ROC and AUC
\end{Description}
%
\begin{Usage}
\begin{verbatim}
tdroc_calc(time, status, marker, times)
\end{verbatim}
\end{Usage}
%
\begin{Arguments}
\begin{ldescription}
\item[\code{time}] Survival time vector

\item[\code{status}] Event indicator (1 = event, 0 = censor)

\item[\code{marker}] Risk score or linear predictor

\item[\code{times}] Vector of time points (e.g., c(365, 730, 1095))
\end{ldescription}
\end{Arguments}
%
\begin{Value}
A data.frame with AUCs for each time
\end{Value}
\HeaderA{validation\_report}{Generate survival model validation report}{validation.Rul.report}
%
\begin{Description}
Generate survival model validation report
\end{Description}
%
\begin{Usage}
\begin{verbatim}
validation_report(
  train_data,
  val_data,
  model,
  times = c(365, 730, 1095),
  time_unit = "days",
  output = "validation_report.pptx"
)
\end{verbatim}
\end{Usage}
%
\begin{Arguments}
\begin{ldescription}
\item[\code{train\_data}] training dataset containing survival outcomes.

\item[\code{val\_data}] validation dataset containing survival outcomes.

\item[\code{model}] fitted 'coxph'

\item[\code{times}] follow-up timepoints

\item[\code{time\_unit}] character label for the time axis (default = "days")

\item[\code{output}] file path (.pptx or .html)
\end{ldescription}
\end{Arguments}
%
\begin{Value}
Writes validation report
\end{Value}
\printindex{}
\end{document}
